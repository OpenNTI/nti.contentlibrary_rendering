\documentclass{book}
%%% Body
\begin{document}

\section{Summary: Introduction to Human Physiology}
\label{reading:intro_physiology}

\subsubsection{Overview of Anatomy and Physiology}

Human anatomy is the scientific study of the body’s structures. In the past, anatomy has primarily been studied via observing injuries, and later by the dissection of anatomical structures of cadavers, but in the past century, computer-assisted imaging techniques have allowed clinicians to look inside the living body. Human physiology is the scientific study of the chemistry and physics of the structures of the body. Physiology explains how the structures of the body work together to maintain life. It is difficult to study structure (anatomy) without knowledge of function (physiology). The two disciplines are typically studied together because form and function are closely related in all living things.

\subsubsection{Structural Organization of the Human Body}

Life processes of the human body are maintained at several levels of structural organization. These include the chemical, cellular, tissue, organ, organ system, and the organism level. Higher levels of organization are built from lower levels.Therefore, molecules combine to form cells, cells combine to form tissues, tissues combine to form organs, organs combine to form organ systems, and organ systems combine to form organisms.

\subsubsection{Functions of Human Life}

Most processes that occur in the human body are not consciously controlled. They occur continuously to build, maintain, and sustain life. These processes include: organization, in terms of the maintenance of essential body boundaries; metabolism, including energy transfer via anabolic and catabolic reactions; responsiveness; movement; and growth, differentiation, reproduction, and renewal.

\end{document}
